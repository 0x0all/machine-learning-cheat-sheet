\chapter{Introduction}

\section{Types of machine learning}
\begin{equation}\nonumber
Machine\, learning\begin{cases}
Supervised\, learning \begin{cases} Classfication\, \\ Regression \end{cases}\\
Unsupervised\, learning \begin{cases} Discovering\, clusters\, \\ Discovering\, latent\, factors\, \\ Discovering\, graph\, structure\, \\ Matrix\, completion \end{cases}\\
\end{cases}
\end{equation}

\section{Three elements of a machine learning method}

\textbf{method = model + strategy + algorithm}

\subsection{Model}
In supervised learning, a model is a decision function or conditional probability distribution  to be learned. The model's hypothesis space contains all possible decition fuctions $f(x)$ or conditional probability distributions $P(y|\vec{x})$.

\subsection{Strategy}
Given a model's hypothesis space, we need a strategy to select which hypothesis is optimal.

\subsubsection{Loss function and risk function}

\begin{definition}
In order to measure how well a function fits the training data, a \textbf{loss function} $L:Y \times Y \rightarrow R \geq 0$ is defined. For training example $(x_i,y_i)$, the loss of predicting the value $\widehat{y}$ is $L(y_i,\widehat{y})$.
\end{definition}

The following is some common loss functions:
\begin{enumerate}
\item 0-1 loss function $L(Y,f(X))=I(Y,f(X))=\begin{cases} 1, & Y=f(X) \\ 0, & Y \neq f(X) \end{cases}$
\item Quadratic loss function $L(Y,f(X))=\left(Y-f(X)\right)^2$
\item Absolute loss function $L(Y,f(X))=\abs{Y-f(X)}$
\item Logarithmic loss function $L(Y,P(Y|X))=-\log{P(Y|X)}$
\end{enumerate}

\begin{definition}
The risk of function $f$ is defined as the expected loss of $f$:
\begin{equation}
R_{exp}(f)=E_p\left[L\left(Y,f(X)\right)\right]=\int _{X \times Y} L\left(y,f(x)\right)P(x,y)dxdy
\end{equation}
which is also called expected loss or \textbf{risk function}.
\end{definition}

\begin{definition}
The risk function $R_{exp}(f)$ can be estimated from the training data as
\begin{equation}
R_{emp}(f)=\dfrac{1}{N}\sum\limits_{i=1}^{N} L\left(y_i,f(x_i)\right)
\end{equation}
which is also called empirical loss or \textbf{empirical risk}.
\end{definition}

You can define your own loss function, but if you're a novice, you're probably better off using one from the literature. There are conditions that loss functions should meet\footnote{\url{http://t.cn/zTrDxLO}}:
\begin{enumerate}
\item They should approximate the actual loss you're trying to minimize. As was said in the other answer, the standard loss functions for classification is zero-one-loss (misclassification rate) and the ones used for training classifiers are approximations of that loss.
\item The loss function should work with your intended optimization algorithm. That's why zero-one-loss is not used directly: it doesn't work with gradient-based optimization methods since it doesn't have a well-defined gradient (or even a subgradient, like the hinge loss for SVMs has).

The main algorithm that optimizes the zero-one-loss directly is the old perceptron algorithm(chapter \S \ref{chap:Perceptron}).
\end{enumerate}

\subsubsection{ERM and SRM}
\begin{definition}
ERM(Empirical risk minimization)
\begin{equation}
\min\limits _{f \in \mathcal{F}} R_{emp}(f)=\min\limits _{f \in \mathcal{F}} \dfrac{1}{N}\sum\limits_{i=1}^{N} L\left(y_i,f(x_i)\right)
\end{equation}
\end{definition}

\begin{definition}
Structural risk
\begin{equation}
R_{smp}(f)=\dfrac{1}{N}\sum\limits_{i=1}^{N} L\left(y_i,f(x_i)\right) +\lambda J(f)
\end{equation}
\end{definition}

\begin{definition}
SRM(Structural risk minimization)
\begin{equation}
\min\limits _{f \in \mathcal{F}} R_{srm}(f)=\min\limits _{f \in \mathcal{F}} \dfrac{1}{N}\sum\limits_{i=1}^{N} L\left(y_i,f(x_i)\right) +\lambda J(f)
\end{equation}
\end{definition}

\subsection{Algorithm}
Namely training algorithm(or learning algorithm), which is used to compute the optimal result according to the strategy. It's a procedural concept.

\section{Cross validation}
\begin{definition}
\textbf{Cross validation}, sometimes called \emph{rotation estimation}, is a \emph{model validation} technique for assessing how the results of a statistical analysis will generalize to an independent data set\footnote{\url{http://en.wikipedia.org/wiki/Cross-validation_(statistics)}}.
\end{definition}

Common types of cross-validation:
\begin{enumerate}
\item K-fold cross-validation. In k-fold cross-validation, the original sample is randomly partitioned into k equal size subsamples. Of the k subsamples, a single subsample is retained as the validation data for testing the model, and the remaining k − 1 subsamples are used as training data.
\item 2-fold cross-validation. Also, called simple cross-validation or holdout method. This is the simplest variation of k-fold cross-validation, k=2.
\item Leave-one-out cross-validation(\emph{LOOCV}). k=M, the number of original samples.
\end{enumerate}

\section{Linear Regression}
Given 
\begin{equation}
\begin{array}{lcl}
\mathcal{D}=\left\{(\vec{x}_i,y_i) | i=1:M\right\} \\
\mathcal{H}=\left\{f(\vec{x_i})=\vec{w}^T\vec{x_i}+b | i=1:M\right\}\\
L(\vec{w},b)=\sum\limits_{i=1}^{M} \left(y_i-f(\vec{x}_i)-b\right)^2\\
\end{array}
\end{equation}

Let $\widehat{\vec{w}}=\left(\vec{w}^T,b\right)^T$,  and
\begin{equation}
\widehat{\vec{X}}=\left(\begin{array}{lcr}
\widehat{\vec{x}}_1^T\\
\widehat{\vec{x}}_2^T\\
\vdots \\
\widehat{\vec{x}}_M^T\\
\end{array}
\right), where\; \widehat{\vec{x}}_i=\left(\vec{x}_i^T,1\right)^T
\end{equation}

We can get
\begin{equation}
\begin{array}{lcr}
L(\widehat{\vec{w}})=\left(\vec{y}-\widehat{\vec{X}}\widehat{\vec{w}}\right)^T\left(\vec{y}-\widehat{\vec{X}}\widehat{\vec{w}}\right)\\
\dfrac{\partial L}{\partial{\widehat{\vec{w}}}}=-2\widehat{\vec{X}}^T\vec{y}+2\widehat{\vec{X}}^T\widehat{\vec{X}}\widehat{\vec{w}}=0\\
\widehat{\vec{X}}^T\vec{y}=\widehat{\vec{X}}^T\widehat{\vec{X}}\widehat{\vec{w}}\\
\widehat{\vec{w}}=\left(\widehat{\vec{X}}^T\widehat{\vec{X}}\right)^{-1}\widehat{\vec{X}}^T\vec{y}
\end{array}
\end{equation}

If $\widehat{\vec{X}}^T\widehat{\vec{X}}$ is singular, the pseudo-inverse can be used, or else the technique of ridge regression described below can be applied.
